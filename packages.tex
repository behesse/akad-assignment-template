\usepackage[automark,headsepline]{scrlayer-scrpage}
\usepackage{geometry}
\usepackage{setspace}
\usepackage{newtxtext,newtxmath}

\geometry{a4paper, left=25mm, right=20mm, top=30mm, bottom=30mm}
\clearpairofpagestyles 			% Seitenstil löschen
\pagestyle{scrheadings}			% Seitenstil mit Kopf- und Fußleiste
\cfoot[\pagemark]{\pagemark}	% Seitenzahl mittig in der Fußleiste
\rohead{\headmark}				% Kolumnentitel rechts in der Kopfleiste

\setkomafont{disposition}{\normalfont\bfseries}
\renewcommand{\familydefault}{\rmdefault}


% Sprache und Kodierung
\usepackage[utf8]{inputenc}
\usepackage[T1]{fontenc}
\usepackage[ngerman]{babel}


% Literatur
\usepackage[style=alphabetic]{biblatex}
\ifdefined\varRemoteBib
   \addbibresource[location=remote]{\varRemoteBib}
\else
   \addbibresource{literature.bib}
\fi


% Gloassar und Abkürzungsverzeichnis
\usepackage[toc, acronym, nopostdot]{glossaries}
\glossarystyle{super}


% Abbildungen aller Art
\usepackage{graphicx}


% Quellcode und Verzeichnis

\usepackage[newfloat]{minted} % Einbinden von minted
\usemintedstyle{bw} % Quellcode Darstellung in schwarz-weiß
\newmintinline{tex}{breaklines,breakbytokenanywhere,breakbefore=\{,fontsize=\small}
\usepackage{caption} % Paket zur Konfiguration von Unterschriften
\usepackage{chngcntr} % Paket zur Konfiguration von Listen-Zählern
\counterwithout{listing}{chapter} % Listen-Zähler ohne Kapitelangabe
\counterwithout{figure}{chapter} % Listen-Zähler ohne Kapitelangabe
\newenvironment{code}{\captionsetup{type=listing}}{} % Erstellung einer Liste für code-Blöcke
\SetupFloatingEnvironment{listing}{listname={Quellcodeverzeichnis}, name={Quellcode}, fileext=lol} % Definition von Bezeichnungen und Verzeichnis für code-Blöcke

% Verlinkungen
\usepackage[
	pdftitle={\varTitle},
	pdfauthor={\varAuthor},
	hyperfootnotes=false,
	colorlinks=true,
	linkcolor=black,
	urlcolor=black,
	citecolor=black
]{hyperref}

\usepackage{csquotes}