\documentclass[
   draft=false,
   paper=a4,
   paper=portrait,
   pagesize=auto,
   fontsize=12pt,
   version=last,
   ngerman,
   parskip,
   numbers=noenddot,
   bibliography=totoc,
   listof=totoc
]{scrreprt}

\input{config.tex}
\usepackage[automark,headsepline]{scrlayer-scrpage}
\usepackage{geometry}
\usepackage{setspace}
\usepackage{newtxtext,newtxmath}

\geometry{a4paper, left=25mm, right=20mm, top=30mm, bottom=30mm}
\clearpairofpagestyles 			% Seitenstil löschen
\pagestyle{scrheadings}			% Seitenstil mit Kopf- und Fußleiste
\cfoot[\pagemark]{\pagemark}	% Seitenzahl mittig in der Fußleiste
\rohead{\headmark}				% Kolumnentitel rechts in der Kopfleiste

\setkomafont{disposition}{\normalfont\bfseries}
\renewcommand{\familydefault}{\rmdefault}


% Sprache und Kodierung
\usepackage[utf8]{inputenc}
\usepackage[T1]{fontenc}
\usepackage[ngerman]{babel}


% Literatur
\usepackage[style=alphabetic]{biblatex}
\ifdefined\varRemoteBib
   \addbibresource[location=remote]{\varRemoteBib}
\else
   \addbibresource{literature.bib}
\fi


% Gloassar und Abkürzungsverzeichnis
\usepackage[toc, acronym, nopostdot]{glossaries}
\glossarystyle{super}


% Abbildungen aller Art
\usepackage{graphicx}


% Quellcode und Verzeichnis

\usepackage[newfloat]{minted} % Einbinden von minted
\usemintedstyle{bw} % Quellcode Darstellung in schwarz-weiß
\newmintinline{tex}{breaklines,breakbytokenanywhere,breakbefore=\{,fontsize=\small}
\usepackage{caption} % Paket zur Konfiguration von Unterschriften
\usepackage{chngcntr} % Paket zur Konfiguration von Listen-Zählern
\counterwithout{listing}{chapter} % Listen-Zähler ohne Kapitelangabe
\counterwithout{figure}{chapter} % Listen-Zähler ohne Kapitelangabe
\newenvironment{code}{\captionsetup{type=listing}}{} % Erstellung einer Liste für code-Blöcke
\SetupFloatingEnvironment{listing}{listname={Quellcodeverzeichnis}, name={Quellcode}, fileext=lol} % Definition von Bezeichnungen und Verzeichnis für code-Blöcke

% Verlinkungen
\usepackage[
	pdftitle={\varTitle},
	pdfauthor={\varAuthor},
	hyperfootnotes=false,
	colorlinks=true,
	linkcolor=black,
	urlcolor=black,
	citecolor=black
]{hyperref}

\usepackage{csquotes}

\pagenumbering{Roman}

\makeglossaries % Glossar generieren

\newglossaryentry{tex}
{
        name=TeX,
        description={Eine Beschreibungssprache für Dokumente}
}

\newglossaryentry{latex}
{
        name=LaTeX,
        description={Erweiterung für TeX}
}

\newglossaryentry{minted}
{
        name=minted,
        description={LaTeX Paket zur Darstellung von Quellcode, basiert auf Pygments}
}

\newglossaryentry{pygments}
{
        name=Pygments,
        description={Python Programm zur Darstellung von Quellcode}
}

\newglossaryentry{newfloat}
{
        name=newfloat,
        description={LaTeX Paket zur Erstellung von Textfluss-Umgebungen wie bei Abbildungen oder Tabellen}
}

\newglossaryentry{zotero}
{
        name=Zotero,
        description={Programm zur Verwaltung von Literatur}
}

\newglossaryentry{better-bibtex}
{
        name=better-bibtex,
        description={Plugin für Zotero, das die Generierung von Zitier-Schlüsseln und das Abrufen eines Literaturverzeichnisses per API ermöglicht}
}

\newglossaryentry{graphicx}
{
        name=graphicx,
        description={Plugin für Zotero, das die Generierung von Zitier-Schlüsseln und das Abrufen eines Literaturverzeichnisses per API ermöglicht}
}



\newacronym{gcd}{GCD}{Greatest Common Divisor}

\newacronym{lcm}{LCM}{Least Common Multiple}

\begin{document}

\begin{titlepage}
\newgeometry{left=20mm, right=20mm, top=35mm, bottom=35mm}
\begin{center}
\thispagestyle{empty}

\Large{\textbf{\varTitle}}
\vspace{1cm}
\onehalfspacing

\large{\varType}

\vspace{1cm}
\normalsize

von

\vspace{.5cm} 
\large{\varAuthor}
\normalsize
\vspace{1cm}

Im Studiengang \varCourse \\
an der staatlich anerkannten AKAD Hochschule Stuttgart
\vspace{2cm}

\today

\vspace{2cm}

\includegraphics[scale=0.35]{akad_logo.png}

\end{center}

\vfill
\begin{spacing}{1.2}
    \begin{tabbing}
        \hspace{9cm} \= \kill
	\textbf{Bearbeitungszeit} \> \varTime \\
	\textbf{Betreuer} \> \varSupervisor \\
	\textbf{Immatrikulationsnummer} \> \varMatriculation \\
	\textbf{E-Mail} \> \href{mailto:\varMail}{\varMail} \\
	\textbf{Adresse} \> \varStreet \\
            \> \varCity
    \end{tabbing}
\end{spacing}
\restoregeometry
\end{titlepage}


\begin{spacing}{1.0} % Verzeichnisse werden mit einzeiligem Abstand gesetzt

\tableofcontents
\printglossary[type=\acronymtype]
\printglossary[toctitle=Glossar]
%\listoffigures 
%\listoftables
\listoflistings

\end{spacing} 
\clearpage

\newcounter{romanPagenumber} 
\setcounter{romanPagenumber}{\value{page}} % Roemische Seitenanzahl speichern.

\pagenumbering{arabic}

\begin{spacing}{1.5}

\chapter{Einleitung}
\section{LaTeX Vorlage für Assignments}
\gls{latex} ist eine Erweiterung des Textverarbeitungssystems \gls{tex}, das von Donald Knuth zwischen 1977 und 1986 entwickelt wurde.\gls{tex} selbst ist eine Beschreibungssprache mit deren Hilfe aus Quellcode bestehend aus Text und Kommandos ein Textdokument generiert wird \autocite{ochsnerTextverarbeitungssystemLaTeX2015}.

\section{Konfiguration}
Die Datei \textit{config.tex.example} bearbeiten und in \textit{config.tex} umbenennen.

\subsection{Literaturverzeichnis aus Zotero}
Der Inhalt des Literaturverzeichnisses kann optional direkt aus \gls{zotero} übernommen werden. Das wird möglich durch das \textit{\gls{better-bibtex}} Plugin für Zotero auf der einen Seite und die Nutzung des Befehls \texinline|\addbibresource[location=remote]{http://[...]/library.biblatex}| auf der Anderen. Auf diese Weise muss keine Literaturverzeichnis-Pflege mehr direkt im \gls{latex} Dokument erfolgen.

\section{Einbinden von Elementen}
\subsection{Grafiken}
Das Template bindet bereits das Paket \textit{\gls{graphicx}} ein. Mit diesem können Grafiken mit dem Kommando \texinline|\includegraphics{grafik.png}| integriert werden. Um im Inhaltsbereich eine Grafik mit Bildunterschrift und Nummerierung einzubinden, die dann auch im Abbildungsverzeichnis referenziert wird, kann beispielhaft folgender Code verwendet werden.
\begin{code}
    \begin{minted}[fontsize=\small,breaklines]{tex}
\begin{figure}[h]
    \centering
    \includegraphics{akad_logo.png}
    \caption{AKAD Logo}
\end{figure}
    \end{minted}
    \caption{Einbindung von Grafiken}
\end{code}

\subsection{Quellcode}
Die Einbindung von Quellcode kann über \textit{\gls{minted}} auf verschiedenen Wegen erfolgen. Zum Einen können ganze Quellcode-Blöcke erstellt und in einem Verzeichnis referenziert werden. Zum Anderen bietet das Paket auch die Möglichkeit, Quellcode in den Textfluss einzubauen.

\begin{code}
    \begin{minted}[fontsize=\small,breaklines,escapeinside=||]{tex}
|\textbackslash|begin{code}
|\textbackslash|begin{minted}[fontsize=\small,breaklines]{html}
        <b>Hello World</b>
        |\textbackslash|end{minted}
    |\textbackslash|caption{Quellcode-Unterschrift}
|\textbackslash|end{code}      
    \end{minted}
    \caption{Einbinden von Quellcode-Blöcken}
\end{code}

Inline Code kann hingegen wie folgt direkt in den Textfluss integriert werden:
\begin{code}
    \begin{minted}[fontsize=\small,breaklines]{tex}
Das ist ein \mintinline{html}|<b>Inline Code</b>|.
    \end{minted}
    \caption{Einbinden von Inline-Quellcode}
\end{code}
\end{spacing}
\clearpage

\pagenumbering{Roman}
\setcounter{page}{\theromanPagenumber}
\printbibliography[title=Literaturverzeichnis]
\clearpage
\end{document}
