\makeglossaries % Glossar generieren

\newglossaryentry{tex}
{
        name=TeX,
        description={Eine Beschreibungssprache für Dokumente}
}

\newglossaryentry{latex}
{
        name=LaTeX,
        description={Erweiterung für TeX}
}

\newglossaryentry{minted}
{
        name=minted,
        description={LaTeX Paket zur Darstellung von Quellcode, basiert auf Pygments}
}

\newglossaryentry{pygments}
{
        name=Pygments,
        description={Python Programm zur Darstellung von Quellcode}
}

\newglossaryentry{newfloat}
{
        name=newfloat,
        description={LaTeX Paket zur Erstellung von Textfluss-Umgebungen wie bei Abbildungen oder Tabellen}
}

\newglossaryentry{zotero}
{
        name=Zotero,
        description={Programm zur Verwaltung von Literatur}
}

\newglossaryentry{better-bibtex}
{
        name=better-bibtex,
        description={Plugin für Zotero, das die Generierung von Zitier-Schlüsseln und das Abrufen eines Literaturverzeichnisses per API ermöglicht}
}

\newglossaryentry{graphicx}
{
        name=graphicx,
        description={Plugin für Zotero, das die Generierung von Zitier-Schlüsseln und das Abrufen eines Literaturverzeichnisses per API ermöglicht}
}



\newacronym{gcd}{GCD}{Greatest Common Divisor}

\newacronym{lcm}{LCM}{Least Common Multiple}